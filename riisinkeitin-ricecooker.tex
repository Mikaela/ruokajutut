\documentclass[a4paper,twocolumn]{artikel3}
\usepackage[english,finnish]{babel}
\usepackage{noto-serif}
\usepackage{cmap}

\title{Rice-cooker quickstart\\Riisinkeittimien pikaopas}
\date{\today}

\begin{document}
\maketitle

\selectlanguage{english}

\section*{Items required}

\begin{itemize}
\item A rice cooker
\item A cup
\item A strainer
\item Rice
\item Salt
\end{itemize}

\section*{Instructions}

Put rice into a cup, pour the cup into a strainer and wash the rice by
putting water into the strainer.

Then put the rice to the rice cooker bowl, add two cups of water there and
one and half tea spoon of salt. Fill the bottom of the rice cooker with
water until it's covered and put it on.

(Boiling water from an electric kettle has been used, but it may be
optional)

Put a timer on for 20 minutes or read instructions from the rice.

When finished, the rice may stay in the cooker if more food things to eat
are coming soon.


\newpage
\selectlanguage{finnish}

\section*{Tarvittavat asiat}

\begin{itemize}
\item Riisinkeitin
\item Kuppi
\item Siivilä
\item Riisi
\item Suola
\end{itemize}


\section*{Ohjeet}

Täytä kuppi riisillä, kaada kuppi siivilään ja pese riisi valuttamalla kuumaa vettä siivilän läpi.

Kaada riisi riisinkeittimen kulhoon, lisää sinne kaksi kuppia vettä ja puolitoista lusikallista suolaa. Täytä riisinkeittimen pohja vedellä ja laita virta päälle.

(Vedenkeittimestä otettua kiehuvaa vettä on käytetty, mutta se voi olla vapaaehtoista.)

Aseta ajastin 20 minuutiksi tai lue riisin ohjeet.

Kun valmis, riisi voi pysyä keittimessä hetken, jos lisää syötävää on tulossa pian.

\end{document}
